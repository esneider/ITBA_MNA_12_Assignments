\documentclass[%
    %draft,
    %submission,
    %compressed,
    final,
    %
    %technote,
    %internal,
    %submitted,
    %inpress,
    reprint,
    %
    %titlepage,
    notitlepage,
    %anonymous,
    narroweqnarray,
    inline,
    twoside,
    invited
    ]{ieee}

\usepackage[utf8]{inputenc}
\usepackage[spanish]{babel}
\usepackage{graphicx}
\usepackage{verbatim}
\usepackage{moreverb}
\usepackage{amsmath}
\usepackage{amsfonts}
\usepackage{amssymb}
\usepackage{fancybox}
\usepackage{float}
\usepackage{fancyvrb}
\usepackage{subfigure}

\newcommand{\latexiie}{\LaTeX2{\Large$_\varepsilon$}}

%\usepackage{ieeetsp}    % if you want the "trans. sig. pro." style
%\usepackage{ieeetc}    % if you want the "trans. comp." style
%\usepackage{ieeeimtc}    % if you want the IMTC conference style

% Use the `endfloat' package to move figures and tables to the end
% of the paper. Useful for `submission' mode.
%\usepackage {endfloat}

% Use the `times' package to use Helvetica and Times-Roman fonts
% instead of the standard Computer Modern fonts. Useful for the 
% IEEE Computer Society transactions.
%\usepackage{times}
% (Note: If you have the commercial package `mathtime,' (from 
% y&y (http://www.yandy.com), it is much better, but the `times' 
% package works too). So, if you have it...
%\usepackage {mathtime}

% for any plug-in code... insert it here. For example, the CDC style...
%\usepackage{ieeecdc}

\begin{document}

%----------------------------------------------------------------------
% Title Information, Abstract and Keywords
%----------------------------------------------------------------------
\title[PageRank]{%
       PageRank}

% format author this way for journal articles.
% MAKE SURE THERE ARE NO SPACES BEFORE A \member OR \authorinfo
% COMMAND (this also means `don't break the line before these
% commands).
\author[Sneidermanis, Sturla]{Darío Sneidermanis, Martín Sturla\\
\\\textbf{19 de Abril de 2012}
}



\journal{Cátedra\ \ Métodos\ Numéricos\ Avanzados,\ ITBA\ }
\titletext{-\ 19, ABRIL\ 2012}
\lognumber{}
\pubitemident{}
\loginfo{19 de Abril, 2012.}
\firstpage{1}

\confplacedate{Buenos Aires, Argentina, 19 de Abril, 2012}

\maketitle               

\begin{abstract} 
Este documento busca analizar distintas implementaciones de PageRank y comparar su desempeño. (bla, terminemoslo despues).
\end{abstract}

\begin{keywords}
PageRank, motores de búsqueda, cadenas de Markov, método de las potencias, autovalores, autovectores.
\end{keywords}

%----------------------------------------------------------------------
% SECTION I: Introduccion%----------------------------------------------------------------------
\section{Introducción}
Con el inicio de las telecomunicaciones masivas por Internet a principio de los 90s y su rápido crecimiento, nace la necesidad de 
crear un índice de los sitios web. Estos índices son usados por los motores de búsqueda, y acompañaron el crecimiento del tráfico 
de Internet. En 1994, World Wide Web Worm poseía un índice de unos 110.000 sitios, y este último era consultado unas 1500 veces diariamente. Ya en 
1997, los motores de búsqueda contaban con índices con una cantidad de sitios que variaba desde 2 hasta 100 millones, y algunos eran consultados
unas  20 millones de veces diariamente. Estos índices eran mantenidos por humanos y por lo tanto, además de ser posiblemente subjetivos, 
no parecían ser suficientemente escalables para poder acompañar el rápido crecimiento del tráfico por Internet. \\
En 1998, habiendo previsto índices en el orden 
de los miles de millones de sitios para el fin del milenio, dos estudiantes de Stanford, Sergey Brin y Lawrance Page, publican un documento llamado 
\textit{"The Anatomy of a Large-Scale Hypertextual Web Search Engine"} que explaya el problema mencionado anteriormente y ofrece una solución; un 
algoritmo computable para indexar sitios web según su importancia, conocido como \textit{PageRank}. Estos estudiantes crearon un motor de búsqueda 
 llamado \textit{Google} utilizando este algoritmo. \\
El motivo de este documento, que busca analizar distintas implementaciones de \textit{PageRank}  y comparar su desempeño, nace del interés de analizar 
teoremas y conceptos antiguos de la matemática, como lo son los autovalores y autovectores, aplicados en un área moderna y con aún mucho por investigar 
como es la informática. (something else?)

%----------------------------------------------------------------------
% SECTION II: Que es PageRank?
%----------------------------------------------------------------------

\section{PageRank}

\subsection{Modelado del problema}

\par El conjunto de los sitios web puede ser representado como un grafo, donde cada sitio es un nodo. Existe un arco guiado entre dos nodos si 
desde el primer sitio hay un hipervínculo al segundo. A cada uno de los nodos se le asigna un valor representando la probabilidad de que 
una cierta persona navegando por Internet esté observando dicho sitio, donimado el \textit{PageRank}  del sitio. 
En cada época o iteración, dicha persona puede decidir quedarse y jamás volver 
a usar un hipervínculo o desplazarse a otro sitio por medio de un hipervínculo.  La probabilidad de que no suceda el 
 primer suceso se denomina factor de \textit{damping}. Nótese que como el primer suceso implica que la persona no navegará nuevamente, 
el fenómeno de \textit{damping} no puede ser representado como un arco al mismo nodo.\\
En vista de dichas propiedades, el grafo puede ser visto como una cadena de Markov, con ciertas propiedades. Se asume que la probabilidad iniciales de 
estar en cualquier nodo es uniforme. Asimismo, se asume que la probabilidad de abrir cualquier hipervínculo de un mismo nodo también es uniforme. Además no existen 
estados absorbentes; si un sitio no tiene hipervínculos, se asume que un usuario podría ingresar el nombre de un nuevo sitio en el navegador, por lo que 
se generan arcos desde dicho estado absorbente a todos los estados del grafo. Debido a que un usuario podría hacer esto incluso en estados 
que no son abosrbentes, se suele agregar dichos arcos en todos los nodos, con una probabilidad residual igual al factor de \textit{damping}.\\
A medida que se itera dicha cadena de Markov y se aproxima al infinito, los valores asociados a cada nodo convergen a un cierto valor. Dicho valor 
representa el peso o importancia de la página. (Necesita gráficos)
\subsection{Marco algebraico}
Para empezar, se debe calcular el valor \textit{PageRank} de un nodo $p_{k}$. Dicho valor se calcula con la suma de la probabilidad de que el usuario 
decida quedarse indefinidamente en el nodo, y aquella asignada a que el usuario haya llegado al sitio a través de un hipervínculo. La probabilidad 
de haber navegado por un hipervínculo, es decir el factor de damping, es representado con una $d$ y empíricamente se ha establecido un valor
de $0,85$ como adecuado. El grado de un nodo $p_{i}$ se denomina $L(p_{i})$ y es la cantidad de hipervínculos que posee. El valor $N$ representa 
la cantidad total de nodos. Se denomina $M_{p_{i}}$ al conjunto de nodos directamente alcanzables desde $p_{i}$. Es decir: (deberiamos derivar mas esto?)\\
\begin{equation}\label{pranknode}
PR(p_{k}) = \frac{1-d}{N} + d\left(\sum_{p_{i}\in M_{p_{i}}}\frac{PR(p_{i})}{L(p_{i})}\right)
\end{equation}
\par Aplicando la ecuación \eqref{pranknode} a todos los nodos y exigiendo que el \textit{PageRank} de cada nodo converja a un valor, 
se obtiene una nueva ecuación en forma matricial:
\begin{equation}\label{pranksolution}
R = \begin{pmatrix}
(1-d)/N \\ (1-d)/N \\ \vdots \\ (1-d)/N
\end{pmatrix} + dMR
\end{equation}
Donde $M$ es la matrix de adyacencia del grafo:
\begin{equation}\label{M}
M=\begin{pmatrix}
l(p_{1},p_{1}) & l(p_{1},p_{2}) & \cdots & l(p_{1},p_{N}) \\
l(p_{2},p_{1}) & \ddots & & \vdots \\
\vdots & & l(p_{i}, p_{j}) & \\
l(p_{N},p_{1}) & \cdots &  & l(p_{N},p_{N}) 
\end{pmatrix}
\end{equation}
Donde $l(p_{i},p_{j})$ es el peso del arco del nodo $p_{i}$ al nodo $p_{j}$, definido como $0$ si dicho arco no existe.\\
La solución $R$, vector representando los valores a los cuales han convergido los nodos, es el \textit{PageRank} final de cada nodo.
\subsection{Método de las potencias en PageRank}
El problema de \textit{PageRank} puede ser resuelto utilizando el método de las potencias. Considérese $\widehat{M}$ definida como:
\begin{equation}\label{mhat}
\widehat{M} = M + \frac{1-d}{N}E
\end{equation}
Donde $E$ es una matriz con todos sus valores iguales a $1$. Es fácil (es? quizás habría que deducirlo un poco) ver que combinando \eqref{pranksolution} 
y \eqref{mhat} se obtiene:
\begin{equation}\label{eigenr}
R = \widehat{M}R
\end{equation}
De la ecuación \eqref{eigenr} se puede deducir que la solución $R$ no es más que el autovector dominante con autovalor $1$ de la matriz de adyacencia 
modificada $\widehat{M}$. Este vector existirá siempre y cuando $det\left|\widehat{M}-I\right| \neq 0 $ (pasa siempre?dem?). Dado que las filas y columnas de 
$\widehat{M}$ suman $1$ (es estocástica), por círculos de Gershgorin el módulo de todos los autovalores deberá ser menor o igual a $1$, por lo que el autovector 
es dominante. El autovalor será único asumiendo que existe una única distribución que satisface la ecuación \label{eigenr}, lo cual es equivalente a exigir 
que $\widehat{M}$ sea inversible. (estaria bueno tirar por aca tipo asumiendo que la cadena de markov es logica bajo los conceptos de page rank bla bla
 estas cosas raras no pueden pasar etc etc).
Por lo tanto, se puede calcular con el método de las potencias. Cabe destacar que como 
$R$ es una distribución de probabilidades, el autovector debe estar normalizado.\\
El método de las potencias 
consiste en tomar un vector $v(0)$ inicializado 
arbitrariamente e iterar según la regla:
\begin{equation}
v(t+1) = \widehat{M}v(t)
\end{equation} 
Normalizando $v(t+1)$ en cada paso, hasta que se cumpla:
\begin{equation}
\left| v(t+1) - v(t) \right| < \epsilon
\end{equation}
Donde $\epsilon$ es un error adecuado. El método de las potencias converge al autovalor dominante siempre y cuando la proyección del vector $v(0)$ a este 
último no sea nula.

%----------------------------------------------------------------------
% The bibliography. This bibliography was generated using the following
% two lines:
%\bibliographystyle{IEEEbib}
%\bibliography{ieeecls}
% where, the contents of the ieeecls.bib file was:
%
%@book{lamport,
%        AUTHOR = "Leslie Lamport",
%         TITLE = "A Document Preparation System: {\LaTeX} User's Guide
%                  and Reference Manual",
%       EDITION = "Second",
%     PUBLISHER = "Addison-Wesley",
%       ADDRESS = "Reading, MA",
%          YEAR = 1994,
%          NOTE = "Be sure to get the updated version for \LaTeX2e!"
%}
%
%@book{goossens,
%        AUTHOR = "Michel Goossens and Frank Mittelbach and
%                  Alexander Samarin",
%         TITLE = "The {\LaTeX} Companion",
%     PUBLISHER = "Addison-Wesley",
%       ADDRESS = "Reading, MA",
%          YEAR = 1994,
%}
%
% The ieeecls.bbl file was manually included here to make the distribution
% of this paper easier. You need not do it for your own papers.

%\clearpage

%\begin{thebibliography}{1}

%\bibitem{lamport1}
%Fierens, P. (2011),
%\newblock {\em Cuadrados mínimos: repaso},
%\newblock Buenos Aires: Instituto Tecnológico de Buenos Aires.

%\bibitem{lamport1}
%Abdi, H.,
%\newblock {\em  Least-squares: {Encyclopedia for research methods for the social sciences}},
%\newblock Thousand Oaks (CA): Sage. pp, 2003.

%\bibitem{lamport1}
%Farebrother, R.W. (1988),
%\newblock {\em Linear Least Squares Computations, STATISTICS: Textbooks and Monographs}, %\newblock New York: Marcel Dekker.

%\bibitem{lamport1}
%Lipson, M.; Lipschutz, S. (2001),
%\newblock {\em Schaum's outline of theory and problems of linear algebra}, 
%newblock New York: McGraw-Hill, pp. 69–80.


%\end{thebibliography}

%----------------------------------------------------------------------


\clearpage





\end{document}