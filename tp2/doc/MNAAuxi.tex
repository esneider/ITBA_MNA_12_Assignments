
\documentclass[%
	%draft,
	%submission,
	%compressed,
	final,
	%
	%technote,
	%internal,
	%submitted,
	%inpress,
	reprint,
	%
	%titlepage,
	notitlepage,
	%anonymous,
	narroweqnarray,
	inline,
	twoside,
	invited
	]{ieee}

\usepackage[utf8]{inputenc}
\usepackage[spanish]{babel}
\usepackage[center]{caption}
\usepackage{graphicx}
\usepackage{verbatim}
\usepackage{color}
\usepackage{moreverb}
\usepackage{amsmath}
\usepackage{amsfonts}
\usepackage{amssymb}
\usepackage{fancybox}
\usepackage{float}
\usepackage{fancyvrb}
\usepackage{subfigure}
\usepackage{mathtools}

\newcommand{\latexiie}{\LaTeX2{\Large$_\varepsilon$}}

%\usepackage{ieeetsp}	% if you want the "trans. sig. pro." style
%\usepackage{ieeetc}	% if you want the "trans. comp." style
%\usepackage{ieeeimtc}	% if you want the IMTC conference style

% Use the `endfloat' package to move figures and tables to the end
% of the paper. Useful for `submission' mode.
%\usepackage {endfloat}

% Use the `times' package to use Helvetica and Times-Roman fonts
% instead of the standard Computer Modern fonts. Useful for the 
% IEEE Computer Society transactions.
%\usepackage{times}
% (Note: If you have the commercial package `mathtime,' (from 
% y&y (http://www.yandy.com), it is much better, but the `times' 
% package works too). So, if you have it...
%\usepackage {mathtime}

% for any plug-in code... insert it here. For example, the CDC style...
%\usepackage{ieeecdc}

\begin{document}

%----------------------------------------------------------------------
% Title Information, Abstract and Keywords
%----------------------------------------------------------------------
\title[Compresión de Imágenes]{%
Compresión de imágenes mediante \\DCT Quantization}

% format author this way for journal articles.
% MAKE SURE THERE ARE NO SPACES BEFORE A \member OR \authorinfo
% COMMAND (this also means `don't break the line before these
% commands).
\author[Sturla, Sneidermanis]{Martín Sturla, Darío Sneidermanis\\\textit{Estudiantes 
       Instituto Tecnológico de Buenos Aires (ITBA)}\\
\\\textbf{31 de Mayo de 2012}
}



\journal{Cátedra\ de\ Met.\ Num.\ Avanzados,\ ITBA\ }
\titletext{- 31, MAYO\ 2012}
\ieeecopyright{0018--9456/97\$10.00 \copyright\ 2011 ITBA}
\lognumber{}
\pubitemident{}
\loginfo{31 de Mayo, 2012.}
\firstpage{1}

\confplacedate{Buenos Aires, Argentina, 31 de Mayo, 2012}

\maketitle               

\begin{abstract} 
El siguiente paper busca analizar la compresión de imágenes utilizando \textit{DCT Quantization} según 
la especificación JPEG original, comparando el nivel de compresión alcanzado y calidad perdida 
utilizando distintos niveles de cuantización.
\end{abstract}

\begin{keywords}
JPEG, compresión de imágenes, transformada discreta coseno, cuantización.
\end{keywords}

%----------------------------------------------------------------------
% SECTION I: Introduccion%----------------------------------------------------------------------
\section{Introducción}

\par Históricamente la compresión de datos siempre ha sido de notable interés para la informática,  
ya sea para guardar algún archivo en algún dispositivo de capacidad limitada o para enviar un 
archivo por algún enlace utilizando la menor cantidad de banda ancha posible.
 Debido a estos requerimientos, surgen los primeros programas que haciendo uso de técnicas de compresión de datos, 
 por ejemplo \textit{Run Length Encoding} o 
codificación de \textit{Huffman}, comprimen y descomprimen archivos. Algunos ejemplos incluyen 
\textit{Winzip}, lanzado en 1991, o \textit{Winrar}, lanzado en 1993. Sin embargo en el caso de las imágenes, 
debido a la naturaleza de sus datos, existen mejores alternativas para comprimirlas. Ya en el 1992, 
un comité conocido como \textit{Join Photographic Experts Group} comienza a publicar pautas 
para la compresión de imágenes, conocidas como la especificación \textit{JPEG}. 
\par La especificación original de JPEG, en la cual se centra este trabajo, indica que la compresión de las 
imágenes consiste en dos etapas fundamentales: una codificación con pérdida de información, es decir \textit{lossy}, 
seguida de una compresión \textit{lossless}, como por ejemplo las dos ya mencionadas. En esta primer etapa 
existen dos maneras de reducir la información: \textit{chroma subsampling} y \textit{DCT Quantization}. La primera, 
la cual no es analizada en este trabajo, consiste en aprovechar que el ojo humano tiene una menor agudeza 
para diferenciar colores que brillantez o \textit{luma}, por lo cual se reduce la resolución de la croma. La segunda 
consiste en hacer uso de que el ojo humano es capaz de diferenciar cambios de 
luma en zonas grandes, pero no tan tenazmente en zonas pequeñas. Debido a esto, se eliminan estas componentes 
de frecuencia espaciales altas en zonas pequeñas (casi imperceptibles para el ojo) y se suavizan los 
cambios de luma (también se efectúa un proceso similar con la croma). Debido a esta eliminación de información, 
existe mayor redundancia de los datos, por lo que un llamado posterior a un algoritmo de compresión será más 
efectivo.
\par El objetivo primordial del trabajo es analizar la técnica de \textit{DCT Quantization} para distintas 
imágenes y matrices de cuantización, analizando en cada caso la tasa de compresión, el error introducido a la 
imagen recuperada y su distorsión gráfica. La sección 2 explica en profundidad la primer etapa de la compresión. 
La sección 3 trata sobre la segunda etapa de compresión. La sección 4 contiene los resultados.

%----------------------------------------------------------------------
% SECTION II: Marco Teórico
%----------------------------------------------------------------------

\section{Etapa de compresión con pérdida de información}

\subsection{Transformación del espacio de colores}

\par La primer etapa de compresión se centra en el manejo de luma y croma, por lo cual es necesario transformar 
los datos de las imágenes, que normalmente contienen la cantidad de rojo, verde y azul en cada pixel (\textit{RGB}). 
Estos datos son transformados a luma en un canal, $Y^{'}$, y crominancia en dos: $C_B$ y $C_R$. Estos últimos 
 representan la proporción de verde con respecto al azul y rojo, respectivamente. 
 Este cambio es también útil dado que la luma se concentra 
en sólo un canal (en \textit{RGB} está disperso entre los tres). Dado que la luma es el componente más 
importante para el ojo humano, se puede hacer una mayor compresión de datos.

\par Las componentes $Y^{'}C_BC_R$ son una combinación lineal de las componentes $RGB$. Las siguientes ecuaciones 
 muestran más en profundidad la naturaleza de la conversión.

\[
 \begin{pmatrix} Y^{'} \\ C_B \\ C_R \end{pmatrix} = \begin{pmatrix} +0.2990 & +0.5870 & +0.1140 \\
-0.1687 & -0.3313 & +0.5000 \\
+0.5000 & -0.4187 & -0.0813  \end{pmatrix} \begin{pmatrix}R \\ G \\ B\end{pmatrix} + \begin{pmatrix}0 \\ 128 \\ 128\end{pmatrix}
\]

\[
 \begin{pmatrix} R \\ G \\ B \end{pmatrix} = \begin{pmatrix} +1.00000 & +0.00000 & +1.40200 \\
+1.00000 & -0.34414 & -0.71414 \\
+1.00000& +1.77200 & +0.00000  \end{pmatrix}  \begin{pmatrix}Y^{'} \\ C_B - 128\\ C_R - 128\end{pmatrix}
\]

\par El apéndice A demuestra que las transformaciones son inversas. !TODO!!

\subsection{Separación en bloques}

\par Luego de la transformación a los canales $Y^{'}C_BC_R$, la imágen se debe separar en bloques, que representan 
las "pequeñas zonas" en las cuales se desea eliminar las frecuencias altas de luma y croma. El tamaño de los bloques 
depende de la tasa de submuestreo de croma usada, y varía entre cuadrados de 8 pixeles y 16 pixeles. Dado que 
no se utilizó submuestreo, se utilizaron bloques de 8.
\par En el caso de que las dimensiones de la imágen no sean múltiplos del tamaño del bloque, se procede a extenderla 
hasta que los bloques quepan perfectamente. Los nuevos pixeles son llenados utilizando información de sus vecinos más 
inmediatos. Otras alternativas analizadas incluyen llenar los canales de los nuevos pixeles con 0, sin embargo se observaron 
mayores distorsiones en los bordes.

\subsection{Transformación del Coseno Discreta}

El siguiente paso consiste en transformar los valores de los canales en cada bloque a sus 
frecuencias espaciales en dos dimensiones, 
utilizando la transformación del coseno discreta. Esta última consiste en una transformada 
de Fourier discreta utilizando únicamente cosenos. Para una cierta matriz cuadrada, la transformada está 
dada por la ecuación:

\begin{equation}
	G_{u, v} = \sum^{N-1}_{x=0}\sum^{N-1}_{y=0}\frac{\alpha(u)\alpha(v)}{N}g_{x, y}\cos(\frac{\pi}{8}(x+\frac{1}{2})u)cos(\frac{\pi}{8}(y+\frac{1}{2})v)
\end{equation}

Donde $g_{x, y}$ representa el elemento de la fila $x$ y la columna $y$ del bloque de entrada, y $G_{u, v}$ 
el elemento de la fila $u$ y columna $v$ de la matriz de frecuencias de salida.

Análogamente la antitransformada está dada por:

\begin{equation}
	g_{x, y} = \sum^{N-1}_{u=0}\sum^{N-1}_{v=0}\frac{\alpha(u)\alpha(v)}{N}G_{u, v}\cos(\frac{\pi}{8}(x+\frac{1}{2})u)cos(\frac{\pi}{8}(y+\frac{1}{2})v)
\end{equation}

En ambas ecuaciones:

\begin{equation}
	\alpha(k) = \left \{
		\begin{array}{rl}
			\sqrt{1} & \mbox{si } k = 0 \\ 
			\sqrt{2} & \mbox{en otro caso} \\
		\end{array}
		\right.
\end{equation}

\par Nótese que los coeficientes de las frecuencias más bajas se encuentran en los valores bajos de $x$ e $y$, 
es decir en la zona superior 
izquierda de la matriz de frecuencias. Similarmente los coeficientes de las frecuencias más altas se encuentran 
en la zona inferior derecha. Esta transformación concentra la señal de entrada en las frecuencias espaciales más bajas, 
es decir en la zona superior izquierda. En particular, comúnmente el mayor valor en módulo es $G_{0,0}$. 
Este valor representa la frecuencia con valor 0, y por lo tanto es un promedio de los 64 valores de entrada. A este 
valor se lo llama $DC coefficient$ dado que en circuitos el promedio de una señal ( es decir el valor asociado 
a la frecuencia cero) es la corriente efectiva directa ($DC$). Al resto de los valores se los llama $AC$ debido 
a que representan la componente alterna de la corriente.

\par También es importante destacar que los valores de los canales son llevados a un rango 
alrededor del $0$ antes de ser transformados, en el intervalo $[-127,128]$. Esto se logra fácilmente restando 
$127$ a todos los valores de entrada.

\subsection{Cuantización}

\par El paso final de la primer etapa de compresión, y el mayor responsable de la pérdida de información, es la cuantización 
de la matriz de coeficientes de frecuencias. Este paso consiste en dividir cada elemento de dicha matriz por un elemento de una  
matriz de cuantización, y redondear el valor obtenido. 
Las matrices de cuantización sugeridas por la especificación JPEG son las siguientes:

\( Qy = 
\begin{pmatrix}
16 & 11 & 10 & 16 & 24 & 40 & 51 & 61 \\
12 & 12 & 14 & 19 & 26 & 58 & 60 & 55 \\
14 & 13 & 16 & 24 & 40 & 57 & 69 & 56 \\ 
14 & 17 & 22 & 29 & 51 & 87 & 80 & 62 \\ 
18 & 22 & 37 & 56 & 68 & 109 & 103 & 77 \\
24 & 35 & 55 & 64 & 81 & 104 & 113 & 92  \\ 
49 & 64 & 78 & 87 & 103 & 121 & 120 & 101 \\
72 & 92 & 95 & 98 & 112 & 100 & 103 & 99 \\ 
\end{pmatrix}\)
\par
\( Qc = 
\begin{pmatrix}

17 & 18 & 24 & 47 & 47 & 99 & 99 & 99 \\
18 & 21 & 26 & 26 & 66 & 99 & 99 & 99 \\
24 & 26 & 56 & 99 & 99 & 99 & 99 & 99 \\ 
47 & 66 & 99 & 99 & 99 & 99 & 99 & 99 \\ 
99 & 99 & 99 & 99 & 99 & 99 & 99 & 99 \\
99 & 99 & 99 & 99 & 99 & 99 & 99 & 99 \\
99 & 99 & 99 & 99 & 99 & 99 & 99 & 99 \\
99 & 99 & 99 & 99 & 99 & 99 & 99 & 99 \\
\end{pmatrix}\)

La fórmula para obtener los valores es la siguiente:

\begin{equation}
\label{eqCuantizacion}
\mathbf{B}(k,l)=round\left(\frac{\mathbf{A}(k,l)}{\mathbf{Q}_Y(k,l)}\right)
\end{equation}

\par En otras palabras, se divide cada elemento de la matriz de coeficientes de frecuencas por su elemento 
correspondiente de la matriz de cuantización y se redondea. Para la luma se utiliza $\mathbf{Q}_Y(k,l)$ y para 
croma $\mathbf{Q}_c(k,l)$.
\par Nótese que existen varios valores de entrada que son transformados a la misma salida, es decir la relación 
no es inyectiva. Debido a esto, se pierde información. En particular, dado un coeficiente $k$ de la matriz 
de cuantización, existen unos $k$ valores que producirán la misma salida. Ergo, 
valores más grandes de la matriz de cuantización implican una mayor pérdida de información. Si se observan ambas 
matrices con detenimiento, se puede apreciar que los valores son mayores en la zona inferior derecha. Esto 
implica que la mayor pérdida de información ocurre en las frecuencias altas de los canales, que es lo que se 
buscaba en un principio.  Asimismo, debido a que la transformada coseno discerta concentra la señal en la zona 
superior izquierda, y los coeficientes de cuantización son mayores en la zona inferior derecha, el resultado de la 
cuantización tendrá una densidad de ceros considerable en esta última zona.

\section{Codificación entrópica}

\subsection{Reordenamiento y codificación}

\par La segunda etapa de la compresión consiste en aplicar una codificación entrópica a los datos cuantizados. Debido 
a la gran densidad de ceros producida por la transformación y cuantización, este tipo de algoritmos logra 
una compresión más efectiva. La especificación JPEG admite dos algoritmos: la codificación de \textit{Huffman} y 
codificación aritmética. Ambos consisten en sustituir instancias de patrones frecuentes por cadenas de bits 
más cortas, con la diferencia que \textit{Huffman} reemplaza cada patrón de entrada por su código respectivo, 
mientras que la aritmética lo transforma en un numero real. La codificación aritmética por lo general obtiene 
resultados entre 5 y 7\% más compactos, pero por lo general se utiliza \textit{Huffman} debido a problema de patentes 
y su mayor velocidad.
\par Sin importar qué algoritmo se utilice, la gran densidad de ceros se traduce a que existe un patrón extremadamente 
frecuente, el cual puede ser reemplazado por una cadena de bits particularmente corta para reducir el tamaño del 
archivo. Sin embargo, se pueden modificar algunos datos para reducir aún más la tasa de compresión.
\subsubsection{Reordenamiento}

\par Antes de ejecutar el algoritmo de codificación, se pueden reordenar los valores de la matriz cuantificada, 
ordenando los valores según qué frecuencia representan. 

\begin{figure}[H]
\centering
	\includegraphics[scale=0.7]{./img/zig-zag.jpg}
	\caption{Reordenamiento de los valores según su frecuencia}
\label{zigzag}
\end{figure}

\par Debido a que los valores asociados a mayores frecuencias eran mayormente cero, este reordenamiento casuará 
cadenas largas de ceros en los datos antes de codificar. Esto resulta en una codificación más compacta.

\subsubsection{Resta del coeficiente DC}

El valor $DC$ suele ser el mayor valor de cada bloque. Si se pueden acotar los valores de los bloques a un intervalo 
pequeño en un entorno del 0, la codificación sería más eficaz. Para eliminar los valores altos de $DC$, se puede hacer 
uso de que el $DC$ representa el promedio de los canales de un bloque. Asumiendo una imagen normal, es muy posible 
que el promedio de los canales en pixeles adyacentes sea similar. Debido a esto, en vez de guardar el $DC$ se puede 
guardar la diferencia entre el $DC$ actual y el del pixel anterior. Esto reduciría la cantidad de valores 
posibles (pudiendo incrementar incluso la frecuencia de ceros marginalmente).

\subsection{Cálculo de cota superior razonable}

\par Para hallar una cota superior razonable y simple al tamaño del archivo resultante luego de la compresión,
 se puede 
utilizar la frecuencia relativa de valores de canales de pixeles con valor en 0. 
Utilizando una codificación de \textit{Huffman}, 
se le podría asignar a estos pixeles un único bit, en 0. En el peor de los casos, el resto de los valores 
varía uniformemente entre $-127$ y $128$, por lo cual se necesitan 9 bits para almacenarlos (un 1 seguido 
de la representación binaria del número).
\par En la imagen original, se utilizan $24$ bits por pixel, $8$ por cada canal. Asumiendo que $p$ es la frecuencia 
de valores de canales en 0, es fácil verificar que en promedio el valor de un canal necesitará:

\begin{equation}
b_{pixel}=3(p+9(1-p))
\end{equation}

\par Es fácil ver también que para valores de $p$ mayores a $\frac{1}{8}$ ya se usan menos de $24$ bits (nótese sin 
embargo que para valores tan bajos de $p$ seguramente haya una mejor codificación que la asumida; esta última asume 
valores más cercanos a $1$ de $p$, como se observa en la práctica). Con un valor de $0.9$, por ejemplo, esta cantidad 
se reduce a $5,4$ bits.
\par Nótese que esta es simplemente una cota superior razonable. Utilizando restas de coeficientes $DC$ y reordenamientos 
la cantidad de bits seguramente será menor. La estimación no contempla el espacio requerido por la tabla de codificaciones 
que se guarda en el encabezado del archivo. Este se puede despreciar asumiendo que la imagen posee dimensiones considerables.



\section{Procedimiento}


\begin{figure}[H]
\centering
	\includegraphics[scale=0.7]{./images/lenna/lenna128matlab.png}
	\caption{Imagen de Lenna de 128x128 pixeles}
\label{lenna128}
\end{figure}



\begin{center}
	\begin{tabular}{|l || c | c | }
		\hline
		\textbf{Imagen} & \textbf{Tamaño en pixeles} & \textbf{Densidad de 0}\\
		\hline
		\hline
		Lenna512 & 512x512 & \< 0.01\%\\
		Nukem	 & 610x685 &  0.01\% \\
		Eye		& 389x353 & 0.01\% \\
		Landscape & 1280x1024 & 0.07\% \\
		\hline
	\end{tabular}
\end{center}
\begin{center}
Tabla 1: Tamaño de las imágenes de muestra
\end{center}
\vspace{0.4cm}

%----------------------------------------------------------------------
% The bibliography. This bibliography was generated using the following
% two lines:
%\bibliographystyle{IEEEbib}
%\bibliography{ieeecls}
% where, the contents of the ieeecls.bib file was:
%
%@book{lamport,
%        AUTHOR = "Leslie Lamport",
%         TITLE = "A Document Preparation System: {\LaTeX} User's Guide
%                  and Reference Manual",
%       EDITION = "Second",
%     PUBLISHER = "Addison-Wesley",
%       ADDRESS = "Reading, MA",
%          YEAR = 1994,
%          NOTE = "Be sure to get the updated version for \LaTeX2e!"
%}
%
%@book{goossens,
%        AUTHOR = "Michel Goossens and Frank Mittelbach and
%                  Alexander Samarin",
%         TITLE = "The {\LaTeX} Companion",
%     PUBLISHER = "Addison-Wesley",
%       ADDRESS = "Reading, MA",
%          YEAR = 1994,
%}
%
% The ieeecls.bbl file was manually included here to make the distribution
% of this paper easier. You need not do it for your own papers.

\clearpage

\begin{thebibliography}{1}

\bibitem{lamport1}
Fierens, P. (2011),
\newblock {\em Análisis Armónico: Guia 02},
\newblock Buenos Aires: Instituto Tecnológico de Buenos Aires.

\bibitem{lamport1}
Khayam, S. (2003),
\newblock {\em  The Discrete Cosine Transform: {Theory and Application}},
\newblock Michigan State University (MI).

\bibitem{lamport1}
Hsu, H. (1987),
\newblock {\em  Análisis de Fourier},
\newblock Addison - Wesley Iberoamericana (DE).

\bibitem{lamport1}
\newblock {\em  JPGE - http://en.wikipedia.org/wiki/Jpg},
\newblock Wikipedia.

\bibitem{lamport1}
\newblock {\em  http://www.cmlab.csie.ntu.edu.tw/cml/dsp/training/coding/jpeg/jpeg/encoder.htm},
\newblock Wikipedia.




\end{thebibliography}

%----------------------------------------------------------------------
\clearpage

\onecolumn

\section*{Anexo A: Imagenes procesadas}

\begin{figure}[H]
\centering
	\includegraphics[scale=0.5]{./images/lenna/lenna512matlab.png}
	\caption{Imagen de Lenna Original de 512x512 pixeles}
\label{lenna512}
\end{figure}


\begin{figure}[H]
\centering
	\includegraphics[scale=0.5]{./images/lenna/lenna512jpg.png}
	\caption{Interpretación RGB de la imagen de Lenna de 512x512 pixeles procesada por el estandar JPEG}
\label{lenna512jpg}
\end{figure}

\begin{figure}[H]
\centering
	\includegraphics[scale=0.5]{./images/lenna/lenna512proc.png}
	\caption{La imagen de Lenna en 512x512 pixeles luedo de ser procesada y recuperada.}
\label{lenna512proc}
\end{figure}


\begin{figure}[H]
\centering
	\includegraphics[scale=0.4]{./images/landscape/landscapeMatlab.png}
	\caption{Imagen Landscape original}
\label{landscapeOrg}
\end{figure}

\begin{figure}[H]
\centering
	\includegraphics[scale=0.4]{./images/landscape/landscapeFormatoJPG.png}
	\caption{Interpretación RGB de la Imagen Landscape luego de ser procesada.}
\label{landscapeJPG}
\end{figure}

\begin{figure}[H]
\centering
	\includegraphics[scale=0.4]{./images/landscape/landscapeFromJPG.png}
	\caption{La Imagen Landscape luego de ser procesada y recuperada.}
\label{landscapeFromJPG}
\end{figure}

\begin{figure}[H]
\centering
	\includegraphics[scale=0.5]{./images/eye/eyeMatlab.png}
	\caption{La Imagen Eye Original}
\label{eyeOrg}
\end{figure}

\begin{figure}[H]
\centering
	\includegraphics[scale=0.5]{./images/eye/eyejpg.png}
	\caption{Interpretación RGB de la Imagen Eye luego de ser procesada.}
\label{eyejpg}
\end{figure}

\begin{figure}[H]
\centering
	\includegraphics[scale=0.5]{./images/eye/eyeProcesed.png}
	\caption{La Imagen Eye luego de ser procesada y recuperada.}
\label{eyeProcesed}
\end{figure}

\par
Las lineas verdes en la imagen \textbf{Figura \ref{eyeProcesed}} y en la \textbf{Figura \ref{nukemProc}} se deben al pagging de 0 para completar los bloques de 8x8 pixeles.


\begin{figure}[H]
\centering
	\includegraphics[scale=0.5]{./images/nukem/nukemMatlab.png}
	\caption{La Imagen Nukem original}
\label{nukemMatlab}
\end{figure}


\begin{figure}[H]
\centering
	\includegraphics[scale=0.5]{./images/nukem/nukemjpg.png}
	\caption{Interpretación RGB de la Imagen Nukem luego de ser procesada.}
\label{nukemjpg}
\end{figure}

\begin{figure}[H]
\centering
	\includegraphics[scale=0.5]{./images/nukem/nukemProc.png}
	\caption{La Imagen Nukem luego de ser procesada y recuperada.}
\label{nukemProc}
\end{figure}


\section*{Anexo B: Funciones Matlab Principales}
Para el desarrollo de este trabajo se utilizó Software Matemático Matlab.

\subsection{Código para convertir el espacio de color de una imagen de RGB a $Y'C_bC_r$}

\VerbatimInput{./code/matlab/rgb2ycbcr.m}

\newpage
\subsection{Código para convertir el espacio de color de una imagen en $Y'C_bC_r$ a RGB}

\VerbatimInput{./code/matlab/ycbcr2rgb.m}

\subsection{Código para aplicar la Transformada Discreta del Coseno}
\VerbatimInput{./code/matlab/TDC.m}

\subsection{Código para aplicar la Anti Transformada Discreta del Coseno}
\VerbatimInput{./code/matlab/ITDC.m}

\newpage
\subsection{Código para aplicar Cuantización de la Luma de una matriz en el espacio de las frecuencias.}
\VerbatimInput{./code/matlab/CuantizacionQy.m}

\subsection{Código para aplicar Cuantización de la crominancia de una matriz en el espacio de las frecuencias.}
\VerbatimInput{./code/matlab/CuantizacionQc.m}

\subsection{Código para aplicar la Descuantización de la Luma de una matriz en el espacio de las frecuencias.}
\VerbatimInput{./code/matlab/DecuantizacionQy.m}

\subsection{Código para aplicar la Descuantización de la crominancia de una matriz en el espacio de las frecuencias.}
\VerbatimInput{./code/matlab/DecuantizacionQy.m}

\section*{Anexo C: Funciones Matlab Auxiliares}

\subsection{Codigo para implementar un inline if}
\VerbatimInput{./code/matlab/choose.m}

\newpage

\subsection{Codigo para dividir una matriz en bloques}
\VerbatimInput{./code/matlab/divideinblocks.m}

\subsection{Codigo para dividir componentes de una matriz}
\VerbatimInput{./code/matlab/roundDivide.m}
\newpage

\subsection{Codigo para multiplicar componentes de una matriz}
\VerbatimInput{./code/matlab/multiplyRound.m}


\subsection{Codigo para procesar un bloque en el estandar JPEG}
\VerbatimInput{./code/matlab/procBlock.m}

\subsection{Codigo para recuperar un bloque en el estandar JPEG}
\VerbatimInput{./code/matlab/deprocBlock.m}

\subsection{Codigo para transformar una imagen en RGB a JPEG}
\VerbatimInput{./code/matlab/RGBtoJPG.m}

\subsection{Codigo para transformar una imagen en JPG a RGB}
\VerbatimInput{./code/matlab/JPGtoRGB.m}

\end{document}
